\section{问题分析}

\subsection{问题一的分析}

从实际问题到模型建立是一种从具体到抽象的思维过程,问题分析这一部分就是沟通这一过程的桥梁,因为它反映了建模者对于问题的认识程度如何,也体现了解决问题的雏形,起着承上启下的作用,也很能反应出建模者的综合水平。

这部分的内容应包括:题目中包含的信息和条件,利用信息和条件对题目做整体分析,确定用什么方法(并不是具体的算法和模型)建立模型,假如我后面要对数据进行清洗、特征缩放、特征编码然后建模,我不能在这里说我要怎么处理缺失值、异常值,也不能说我要用独热编码、要用数据归一化,更不能说我要用SVM、要用树模型啦,而是应该说“本文首先对数据集进行合理的数据预处理与特征工程,并建立分类模型实现预测……”。一般是每个问题单独分析一小节,分析过程要简明扼要, 不需要放结论。

建议在文字说明的同时用图形或图表(例如流程图)列出思维过程,这会使你的思维显得很清晰,让人觉得一目了然!接下来我给出2023年数学建模国赛C题的“问题分析”示例供大家参考。




\subsection{问题二的分析}

\textbf{针对问题二},第一小问要求分析各蔬菜品类的销售总量与成本加成定价的关系。首先为了便捷分析,将四个附件的数据信息合并;接着按日聚合数据并将销售量并求和,并取销售单价的均值;然后采用孤立森林模型清洗异常值。因数据未提供成本加成定价,所以首先计算出此数值。接着采用多项式回归拟合模型,输出销售总量与成本加成定价间的函数解析式,客观精确地分析出两者关系。

第二小问要求给出使商超受益最大的各品类蔬菜未来一周的的日补货总量和定价策略。首先本文采用三重指数平滑模型预测出计算收益所需要的每日批发价格。接着的通过置信区间得出日补货量与成本加成定价的取值范围。得到取值范围后,将每日收益公式的最大值点求出,从而得到使商超受益最大的的最优补货定价策略



\subsection{问题三的分析}

\textbf{针对问题三},本问要求在给定对蔬菜数量、市场陈列量以及现实问题等约束的情况下制定7月1日的单品补货量以及定价的策略,使商超尽可能收益最大。首先需已知当日的单品批发价,由于仅需一日的数据,故直接取6月30日的单品批发价;接着需确定补货量与定价的约束条件,基于对前一周销售量与成本加成定价的描述性统计分析,并加以扰动生成各单品的约束条件;最后确定目标函数并建立优化模型并对其进行求解。


\subsection{问题四的分析}


\textbf{针对问题四},本问要求给出需要采集的数据并推出这些数据与上题中求出的结果之间的关系。因此首先想出与补货和定价决策可能有关的数据,再将这些数据进行分析,找出最为合适的数学模型来判断所提供数据与原数据之间是否存在相互关联。本文采用了回归分析和时间序列模型,分别构建出商超销量与竞争对手销量关系的公式和市场价与销售价之间的关系对销售量的影响公式来优化补货和定价决策的制定。


刚刚说了大家应该在问题分析部分给一个流程图,当然流程图很大,放在这里太占空间,也不美观,因此我采取的方式是把流程图放在附录,并在问题分析部分作出索引。这里本来应该把关于流程图的文字介绍放在问题分析最后面,并用“如图X”进行索引,索引的数字是蓝色的,评委很容易看出来!如图~\ref{fig3}~。

这里提醒一下啊,\LaTeX 中的所有编号都是自动排序的,不需要管,后面我会告诉大家如何插入图片表格,上面那个“杠ref\{label\}”的意思就是索引某个图表的标签,“fig3”是我给插入在附录的流程图设置的标签,如同姓名一样用作区分而没有实际意义。
























