\section{模型评价和改进}

% 本章内容是要从宏观解释本文模型的厉害之处,本来是可以不写缺点的,一般写2个部分:优点、推广,下面我给的案例是我2022年国赛写的,初生牛犊不怕虎,写了缺点,本来当时是可以写基于Voting的推广的,但是国赛要求的篇幅,当时已经不够用了,所以作罢!


\subsection{模型优点}

\begin{enumerate}
	\item 在问题一中对文物表面是否风化与类型、颜色、纹饰关系分析过程中,不仅对于单变量之间进行了分析,还进一步用树模型进行了多变量与单变量的分析,同时利用互信息进一步对结果进行检验,提高模型的合理性。
	\item 本文做了大量图表来统计分析数据特点,直观的对比出两类玻璃风化前后的化学成分的变化量以及各类玻璃化学成分之间的关系。
	\item 在文本中多次对模型进行调参,利用混淆矩阵和多个指标检验,提高了模型的准确性。
	\item 使用强分类器,构建Voting集成算法,得到一个完美模型,得到结果可信度高。
\end{enumerate}



\subsection{模型缺点}

\begin{enumerate}
	\item 做编码时由于颜色样本有7个非叙述类别,而数据集中存在大量的分类特征,没有找到合理高效的特征编码方式。
\end{enumerate}


% \subsection{模型改进}

% \subsection{模型推广}