\section{问题一的建模与求解(公式与建模部分行文简介)}

\subsection{数学公式}\label{subsection1}  % 这个label代码的意思会在下一章解释


模型建立部分是需要有大量的数学公式的,而很多小伙伴都没有 \LaTeX 基础,这里就不得不提到Axmath或Mathtype了,队友可以先用Axmath或Mathtype在Word或者WPS中敲好公式,然后再用Axmath或Mathtype直接将公式转换为Tex代码。


数学公式的排版大致有两种类型,前者是行内公式,如$\int_{1}^{+\infty}f(x)\textbf{d}x$,这种情况一般适用于公式量不大的时候,一般就把公式当作文字放在文中了;另一种是行间公式,一般长公式、复杂公式就会单独占一行,比如这样:
\begin{equation}
	\displaystyle\int_{a}^{b}f(x)\textbf{d}x = a.
\end{equation}

要注意的是所有行间公式结束的时候都必须要有标点符号(英文版本的),一般就是英文逗号或者英文句号,我是这样判断的:当行间公式后面跟着“其中XXX是XX”这样的话时用逗号,其余情况全是句号。另外如果选用了英文句号,那么行间公式后面的文字段落与“杠end\{equation\}”这一行中间就要空一行表示分段,也就是这是新的一段,但如果使用的是逗号则不能空行,表示这一段还没结束!

有的时候后文需要用到前文的数学公式可以通过交叉引用来避免重复内容,首先在行间公式中加一行“杠label\{name\}”,这个name你可以任意取定但必须全文唯一,比如我下面的数学公式取定为“equation1”,然后我就可以说:如公式~\ref{equation1}~所示:
\begin{equation}
	\displaystyle\int_{a}^{b}f(x)\textbf{d}x = a.
	\label{equation1}
\end{equation}

\LaTeX 会自动给你编号并把数字的颜色设置为蓝色,评委很容易看出来,点击就可以跳转到这个公式!

有的时候可能会遇到特别长的公式,可以仿照公式~\ref{换行公式}~进行换行:
\begin{equation}
	\begin{split}
		y &=(-2.014e-21){{x}^{12}}+(2.417e-18){{x}^{11}}-(1.275e-15){{x}^{10}}+(3.892e-13){{x}^{9}} \\ 
		& \quad   -(7.605e-11){{x}^{8}}+(9.957e-09){{x}^{7}}-(8.872e-07){{x}^{6}}+(5.356e-05){{x}^{5}} \\ 
		&  \quad  -0.002138{{x}^{4}}+0.05374{{x}^{3}}-0.7827{{x}^{2}}+5.619x-0.389.
	\end{split}
	\label{换行公式}
\end{equation}


关于数学公式的问题就介绍这么多。




\subsection{建模部分行文简介}


下面我来介绍一下一个标准的“模型建立与求解(一级标题)”内容应该怎么写。

首先是前面的模型建立部分,我建议大家二级标题这么写:“基于XXX的XXX”,当然有的时候前面还会有数据预处理之类的准备工作。模型建立部分应该主要是数学公式进行解释,毕竟这是数学建模比赛,要求用数学方法、计算机手段解决问题。首先你应该介绍一下你的数据集并给它设一个数学符号,然后将其带入到数学公式进行运算,直到能算出目标,当然这里并不是具体的目标,只是数学意义上的目标。

然后把前面所有的“基于XXX的XXX”写完以后就该谢模型求解了,大家注意:每一个“基于XXX的XXX”都是二级标题,模型求解也是二级标题(反正我是这么做的)。模型求解部分做两件事:把求解出的结果或者结论(比如机器学习模型预测结果、显著性检验的结果)以三线表或可视化的形式给出;把对模型的测试结果给出(比如混淆矩阵、ROC图、MAE、MSE等)。有的同学可能有疑问:那后面的“模型评价(一级标题)”是干嘛的,我觉得那里是整体的评价和推广,不是单个模型哈,当然每个人理解不一样也很正常,如果觉得对就可以相信自己。








