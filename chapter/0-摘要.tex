\begin{abstract}

    摘要是全文最重要的部分,也是阅卷人首先看到的部分,阅卷人会只根据摘要将文章分成三六九等,所以如果不认真写摘要的话你就会有大麻烦,请务必留至少两小时用于摘要的打磨,将其控制在1面!!!
    
    针对问题一:在写摘要的时候请搞清楚,首先摘要的第一段是题设的背景,你需要随便胡扯几句,但别扯太多,毕竟要把摘要控制在一面很难!然后写完第一段后从第二段开始就要开始介绍你对每个题目的理解、过程以及求解与评价,语言尽量精炼,不要啰嗦,再强调一次:那么多内容的摘要压缩在一面非常难!下面我将给大家演示如何对于自己已经完成的“模型建立与求解”部分进行摘要描述。
    
    针对问题二,本文基于可视化和假设检验对所给数据集进行数据分析。首先对产品的销售价格和销售量的关系进行探究,本文分别通过\textbf{斯皮尔曼相关系数}对相关性进行定量描述,求得$\rho$=-0.2946,反映出\textbf{销售价格和销售量的相关性较弱}。接着对区域与销量的关系进行探究,通过\textbf{方差检验}得知\textbf{不同地区对订单的需求量有显著差异},并通过直方图探究出不同区域产品的需求量的不同特性。然后,本文对产品的销售方式与需求量的关系进行探究,通过\textbf{Mann-Whitney U}检验得出\textbf{线上销售与线下销售的销售量存在显著差异}。最后,通过\textbf{单样本Wilcoxon符号秩检验}将时间序列整体与促销活动单日的销量相比较,得出\textbf{促销活动单日的销量与平时的销量有显著差异}。为了将各特征的分布以及定量分析所得出的结论直观化,本文利用小提琴图、箱线图、直方图等对相关特征进行可视化,结果与定量分析一致。
    
    针对问题三,我认为摘要关于每个题的内容应该有以下特点:整段的写作应该是循序渐进的,比如使用“首先本文XXX”、“接着本文XXX”、“然后本文XXX”以及“最后本文XXX”这样的形式;\textbf{关键的方法和结论}应该是加粗显示,\LaTeX 在文中的加粗的代码想必大家已经看到了;段落中除了写建模过程中我们用了什么方式做了什么,还应该描述结论——定性的结论要直接给出,定量的结论如果较多可以概括性描述。
    
    针对问题四,最后说一说关于这个模板,我认为数学建模竞赛的结果与论文的关联性是非常强的,所以不应该一味追求技术的实现,应该尤其重视文章的撰写与排版,\LaTeX 作为一个非常经典的排版工具有一定的上手门槛,如果大家使用好了可以给文章加分不少,但是务必在比赛前多尝试,尤其是把自己之前的论文套入模板看看是否会出问题,否则比赛的时候如果花大量的时间摆弄\LaTeX ,那就是得不偿失了。
    
    \vspace{1em} % 移除2个行距的空白:这句代码千万不能动!!!移除2个行距的空白:这句代码千万不能动!!!
    
    \noindent{\textbf{关键词:}随机森林;方差选择法;Voting Classifier;层次聚类分析;决策树}
    
\end{abstract} 