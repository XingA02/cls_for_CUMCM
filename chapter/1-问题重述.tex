\section{问题重述}

\subsection{问题背景}

数学建模比赛论文是要我们解决一道给定的问题,所以正文部分一般应从问题重述开始,一般确定选题后,写手就可以开始写这一部分了——毕竟这个部分不需要等编程手和建模手掰扯。

这部分的内容是将原问题进行整理,将问题背景和题目分开陈述即可,所以基本没啥难度——问题背景就是把赛题给出的具体背景简要的阐释并加上具体的理解,问题提出就是把3(4)个小题用自己的话复述一遍。

本部分的目的是要吸引读者读下去,所以文字不可冗长,内容选择不要过于分散、琐碎,措辞要精练。
注意:在写这部分的内容时,绝对不可照抄原题!(论文会查重)
应为:在仔细理解了问题的基础上,用自己的语言重新将问题描述一遍。语言需要简明扼要,没有必要像原题一样面面俱到。


\subsection{问题提出}


下面我将给出2022年数学建模国赛C题的“问题提出”示例供大家参考。现有一组由专家给出的关于玻璃的数据,要求通过分析与建模解决下面若干问题:

\textbf{问题1:} 分析这些玻璃文物的表面风化与其玻璃类型、纹饰和颜色的关系以及文物样品表面有无风化化学成分含量的统计规律,并对风化前的化学含量进行分析。

\textbf{问题2:} 依据附件数据分析两种玻璃的分类规律;对于每个类别选择合适的化学成分进行亚类划分,给出划分方法及结果,并分析分类结果的合理性和敏感性。 

\textbf{问题3:} 分析未知类别玻璃文物的化学成分,鉴别其所属类型并对分类结果的敏感性进行分析。 

\textbf{问题4:} 针对不同类别的玻璃文物样品,分析其化学成分之间的关联关系并对其差异性进行比较。